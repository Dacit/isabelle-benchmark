\section{Future Work}\label{sec:future}

If the Isabelle computation model does not change,
there is not much left to be done on the topic of performance prediction:
The benchmark from which Isabelle performance is predicted is widely popular and data for new hardware is usually quickly added,
and the fit is about as good as one can hope for.
Still, the model should be validated after a few years,
and we are curious to see whether future hardware characteristics will be able to break the trend.

One other aspect that we did not touch on is running multiple independent sessions in parallel,
which is often possible on automated large-scale Isabelle builds, e.g., for the Archive of Formal Proofs.
This can be done on multiple processes that run independently of each other and greatly increases the usefulness of larger server CPUs with many cores;
then, other parameters such as memory bandwidth might be more important and have to be analyzed.
However, given the large cost of such machines,
it would be much more economical to instead distribute the build on multiple cheap but fast desktop CPUs
(especially when latency is a concern, not throughput).