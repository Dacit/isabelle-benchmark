% !TeX root = ../main.tex

\section{Results}\label{sec:results}
At the time of writing this paper,
\num{\numRuns} results for a total of \num{\numConf} unique configurations have been reported,
utilizing \num{\numCpus} distinct CPUs.
Those include Intel Desktop/Server CPUs from Sandy Bridge to Alder Lake,
AMD Ryzen Zen2 to Zen4 processors as well as Epyc and Threadripper server systems,
a Fujitsu A64FX, and Apple M1 processors.

% !TeX root = ../main.tex


\DTLloaddb{topcpus}{top_cpus.csv}

\begin{table}[htbp]
  \caption{The five processors with the lowest time to solution, using each processor's optimal configuration (median value taken for multiple runs).}
  \label{tab:top_5_cpus}
  \begin{tabular}{lr}\toprule
    \textbf{Processor Model} & \textbf{Wall-Clock Time}
    \DTLforeach*{topcpus}{\cpu=cpu,\time=time}{
        \DTLiffirstrow{\\\cmidrule{1-2}}{\\}
        \cpu & \SI{\time}{\second}
    }
    \\\bottomrule
  \end{tabular}
\end{table}

Table \ref{tab:top_5_cpus} shows the five CPUs with the lowest time to solution,
using the median value as an aggregate for multiple runs of the same configuration.
Older Intel and AMD consumer hardware is surpassed by the Apple M1 Pro chip;
only the most recent Intel core line performs better.
Due to the nature of the benchmark, server and high performance hardware does not rank highly,
with the best performing system (2x AMD Epyc 7742) clocking in at \SI{\bestServerTime}{\second}.

In the following, we analyze how Isabelle configuration influences performance,
investigate the impact of hardware parameters,
and then compare our results to other computational benchmarks.
Where individual CPUs were concerned,
we filtered out special system configurations (e.g., overclocked hardware, dual-cpu systems, power-saving mode).
We also encountered a small number of extreme outliers where Isabelle run-time was much longer than expected.
For two of those, we could identify the user and investigate;
in both cases, the system configuration was at fault
(excessive swapping, UEFI set to \enquote{silent mode})
and when corrected, results were much closer to the rest of the data.
We could not investigate the third extreme outlier but excluded it from the following,
since it is likely to stem from a similar cause.

\subsection{Multi-Threaded Performance}\label{sec:performance_threads}
The number of threads used plays a major role in the overall performance.
\autoref{fig:time_by_threads} illustrates how the wall clock time and CPU time compare from a single thread to up to \num{128} threads.
The optimal wall-clock time is achieved with \numrange{8}{16} threads
depending on the hardware and greatly increases if more threads are used.
This is typical behavior for a strong-scaling benchmark like ours,
where the relative impact of communication increases with an increase in the number of threads used.
For more than the optimal number of threads,
the run-time increases substantially.
The underlying limitations of the parallel computing model 
-- the single-threaded garbage collection of the Poly/ML run-time and worker starvation after parallelization is saturated --
were already discussed in~\cite{PolyParallel2010Matthews},
albeit tests were run on a machine with 32 cores.
It might be a surprise that the scalability is so low when distributing across more threads.
In contrast,
the CPU time divided by number of threads
(which is not an ideal metric, but the only feasible solution due to the nature of the benchmark)
flattens out at eight threads.
In small-scale experiments, we found that the JVM process takes up a constant amount of CPU time independent of the number of threads (about \SI{26}{\percent} in single-core mode).
This means that there is not too much computation overhead
but the hardware can not be properly utilized by the ML process,
most likely due to the single-threaded garbage collection that stops all threads when running.
This is an inherently sequential task, which means that Amdahl's Law (the speedup is limited by $1 / (1-\text{parallelizable portion})$~\cite{amdahls}) limits the achievable speedup for this problem.
% !TeX root = ../main.tex

\begin{figure}[htbp]
    \centering
    \begin{tikzpicture}
    \begin{groupplot}[
      group style={
        group name=timebythreads,
        group size=2 by 1,
        x descriptions at=edge bottom,
        y descriptions at=edge left,
        horizontal sep=0,
      },
      width=0.5\textwidth,
      height=192pt,
    ]
    \nextgroupplot[
      xmode=log,
      xtick={1,2,4,8,16,32,64,128},
      xlabel={Thread Count (log scale)},
      every axis x label/.append style={at=(ticklabel cs:1.0)},
      x label/.append style={at=(ticklabel cs:1.0)},
      log ticks with fixed point,
      ylabel={Wall-Clock Time},
      y unit={\si{\minute}},
      ytick distance=480,
      ymax=3540,
      ymin=0,
      yticklabel={\pgfmathparse{int(\tick/60)/100}\pgfmathprintnumber[skip 0.=true, dec sep={}, fixed,precision=2,zerofill]{\pgfmathresult}:\pgfmathprintnumber[skip 0.=true, dec sep={}, fixed,precision=2,zerofill]{0.0}},
    ]
    \foreach \cpu in \cpus {
    	\addplot[
    	  discard if not={cpu}{\cpu},
          color=black,
          mark=*,
          mark options={color=black,fill=white,scale=1,opacity=0.9},
          opacity=0.5
        ] table [
          col sep=comma,
          x=threads,
          y=time
        ] {time_by_threads.csv};
    }
    \nextgroupplot[
      xmode=log,
      xtick={1,2,4,8,16,32,64,128},
      log ticks with fixed point,
      ylabel={CPU Time per Thread},
      yticklabel pos=right,
      ylabel near ticks,
      ytick distance=480,
      y unit={\si{\minute}},
      ymax=3540,
      ymin=0,
      yticklabel={\pgfmathparse{int(\tick/60)/100}\pgfmathprintnumber[skip 0.=true, dec sep={}, fixed,precision=2,zerofill]{\pgfmathresult}:\pgfmathprintnumber[skip 0.=true, dec sep={}, fixed,precision=2,zerofill]{0.0}},
    ]
    \foreach \cpu in \cpus {
    	\addplot[
          discard if not={cpu}{\cpu},
          color=black,
          mark=*,
          mark options={color=black,fill=white,scale=1,opacity=0.9},
          opacity=0.5
        ] table [
          col sep=comma,
          x=threads,
          y=cputime
        ] {cputime_by_threads.csv};
    }
    \end{groupplot}
    \end{tikzpicture}
    \caption{Run-time by number of threads (log scale): Wall-clock time on the left, CPU time per thread on the right.}
    \label{fig:time_by_threads}
\end{figure}


The parallel efficiency paints a similar picture in \autoref{fig:parallel_efficiency},
decreasing  almost linearly (on the logarithmic x-axis) up to \num{32} threads
at which it is at a median of \num[round-mode=places,round-precision=3]{\effLin}.
With the number of threads tending to the limit of \num{128},
it approaches \num[round-mode=places,round-precision=3]{\effEnd}.
There is an outlier where the parallel efficiency is over one --
super-linear speedup is unusual but can appear in practice because of caching effects or resource contention in the measured system.
% !TeX root = ../main.tex

\begin{figure}[htbp]
    \centering
    \begin{tikzpicture}
    \begin{axis}[
      xmode=log,
      xtick={1,2,4,8,16,32,64,128},
      xlabel={Thread Count (log scale)},
      log ticks with fixed point,
      yticklabel={\pgfmathprintnumber[zerofill,precision=1,fixed]{\tick}},
      ymin=-0.1,
      ymax=1.1,
      ylabel={Parallel Efficiency},
    ]
    \foreach \cpu in \cpus {
    	\addplot[
    	  discard if not={cpu}{\cpu},
          color=black,
          mark=*,
          mark options={color=black,fill=white,scale=1,opacity=0.9},
          opacity=0.5
        ] table [
          col sep=comma,
          x=threads,
          y=rel
        ] {parallel_efficiency.csv};
    }
    \end{axis}
    \end{tikzpicture}
    \caption{Parallel efficiency by number of threads. The HOL-Analysis build does not scale past a low number of threads, with parallel efficiency already down by half when using eight threads.}
    \label{fig:parallel_efficiency}
\end{figure}



\subsection{Performance Impact of Heap Memory}
As preliminary results indicated that heap memory
(as long as sufficient)
only plays a minor role in performance,
we keep the JVM and Poly/ML processes at the same heap size.
We know from experience that a few gigabytes of memory suffice for HOL-Analysis;
however, increased parallelism requires more memory in principle due to memory overhead.
Hence, the range of examined heap sizes depends on the number of threads used.
\autoref{fig:heap_boxplot} shows the change in run-time for different heap settings relative to the minimal setting.
The boxes capture the \num{25} and \num{75} percentiles as height and sampling size as width;
whiskers correspond to the extreme values.
The results show that performance is not affected very much by heap size.
Following the line of medians,
wall-clock time slightly increases above \SI{16}{\giga\byte}
(where the \SI{64}{\bit} Poly/ML backend needs to be used, as the more efficient $64\_32$ mode does not allow more than \SI{16}{\giga\byte}),
as well as for very large values.
We observed a single outlier for \num{64} threads and \SI{128}{\giga\byte} heap memory
at a relative factor of \num[round-mode=places,round-precision=2]{\heapOutlier}.
% !TeX root = ../main.tex
\makeatletter
\pgfplotsset{
    boxplot prepared from table/.code={
        \def\tikz@plot@handler{\pgfplotsplothandlerboxplotprepared}%
        \pgfplotsset{
            /pgfplots/boxplot prepared from table/.cd,
            #1,
        }
    },
    /pgfplots/boxplot prepared from table/.cd,
        table/.code={\pgfplotstablecopy{#1}\to\boxplot@datatable},
        row/.initial=0,
        make style readable from table/.style={
            #1/.code={
                \pgfplotstablegetelem{\pgfkeysvalueof{/pgfplots/boxplot prepared from table/row}}{##1}\of\boxplot@datatable
                \pgfplotsset{boxplot/#1/.expand once={\pgfplotsretval}}
            }
        },
        make style readable from table=lower whisker,
        make style readable from table=upper whisker,
        make style readable from table=lower quartile,
        make style readable from table=upper quartile,
        make style readable from table=median,
        make style readable from table=lower notch,
        make style readable from table=upper notch,
        make style readable from table=sample size,
        make style readable from table=draw position
}
\makeatother

\begin{figure}[htbp]
\centering
\begin{tikzpicture}
\pgfplotstableread[col sep = comma]{rel_time_by_heap_box.csv}\reltimebyheapbox
\begin{groupplot}[
  group style={
    group name=threads,
    group size=1 by 4,
    vertical sep=0pt,
    x descriptions at=edge bottom,
    y descriptions at=edge left,
  },
  height=120pt,
  width=307pt,
  xmin=2.5,
  xmax=9.5,
  ymin=0.7,
  ymax=1.3,
  xlabel={Heap Memory (log scale)},
  x unit={\si{\giga\byte}},
  boxplot/draw direction=y,
  boxplot/variable width,
  xticklabel={\pgfmathparse{pow(2,\tick)}\pgfmathprintnumber[int trunc]{\pgfmathresult}},
  boxplot/every whisker/.style={gray},
  boxplot/every median/.style={very thick},
  boxplot/draw/median/.code={},
]
\nextgroupplot
  \addplot[forget plot,boxplot prepared from table={table=\reltimebyheapbox,row=1,lower whisker=lw,upper whisker=uw,lower quartile=lq,upper quartile=uq,median=med,sample size=ss,draw position=heap}, boxplot prepared] coordinates {};
  \addplot[forget plot,boxplot prepared from table={table=\reltimebyheapbox,row=2,lower whisker=lw,upper whisker=uw,lower quartile=lq,upper quartile=uq,median=med,sample size=ss,draw position=heap}, boxplot prepared] coordinates {};
  \addplot[forget plot,boxplot prepared from table={table=\reltimebyheapbox,row=3,lower whisker=lw,upper whisker=uw,lower quartile=lq,upper quartile=uq,median=med,sample size=ss,draw position=heap}, boxplot prepared] coordinates {};
  \addplot[
    densely dotted,
    mark=-,
    mark size={3pt},
    mark options={solid,very thick},
    discard if not={threads}{8},
  ] table [
    col sep=comma,
    x=heap,
    y=med,
  ] {rel_time_by_heap_box.csv};
  \addlegendentry{medians};
\nextgroupplot
  \addplot[boxplot prepared from table={table=\reltimebyheapbox,row=5,lower whisker=lw,upper whisker=uw,lower quartile=lq,upper quartile=uq,median=med,sample size=ss,draw position=heap}, boxplot prepared] coordinates {};
  \addplot[boxplot prepared from table={table=\reltimebyheapbox,row=6,lower whisker=lw,upper whisker=uw,lower quartile=lq,upper quartile=uq,median=med,sample size=ss,draw position=heap}, boxplot prepared] coordinates {};
  \addplot[boxplot prepared from table={table=\reltimebyheapbox,row=7,lower whisker=lw,upper whisker=uw,lower quartile=lq,upper quartile=uq,median=med,sample size=ss,draw position=heap}, boxplot prepared] coordinates {};
  \addplot[boxplot prepared from table={table=\reltimebyheapbox,row=8,lower whisker=lw,upper whisker=uw,lower quartile=lq,upper quartile=uq,median=med,sample size=ss,draw position=heap}, boxplot prepared] coordinates {};
  \addplot[
    densely dotted,
    mark=-,
    mark size={3pt},
    mark options={solid,very thick},
    discard if not={threads}{16},
  ] table [
    col sep=comma,
    x=heap,
    y=med,
  ] {rel_time_by_heap_box.csv};
\nextgroupplot[ylabel={Relative Run-Time},every axis y label/.append style={at=(ticklabel cs:1.0)}]
  \addplot[boxplot prepared from table={table=\reltimebyheapbox,row=10,lower whisker=lw,upper whisker=uw,lower quartile=lq,upper quartile=uq,median=med,sample size=ss,draw position=heap}, boxplot prepared] coordinates {};
  \addplot[boxplot prepared from table={table=\reltimebyheapbox,row=11,lower whisker=lw,upper whisker=uw,lower quartile=lq,upper quartile=uq,median=med,sample size=ss,draw position=heap}, boxplot prepared] coordinates {};
  \addplot[boxplot prepared from table={table=\reltimebyheapbox,row=12,lower whisker=lw,upper whisker=uw,lower quartile=lq,upper quartile=uq,median=med,sample size=ss,draw position=heap}, boxplot prepared] coordinates {};
  \addplot[boxplot prepared from table={table=\reltimebyheapbox,row=13,lower whisker=lw,upper whisker=uw,lower quartile=lq,upper quartile=uq,median=med,sample size=ss,draw position=heap}, boxplot prepared] coordinates {};
  \addplot[boxplot prepared from table={table=\reltimebyheapbox,row=14,lower whisker=lw,upper whisker=uw,lower quartile=lq,upper quartile=uq,median=med,sample size=ss,draw position=heap}, boxplot prepared] coordinates {};
  \addplot[
    densely dotted,
    mark=-,
    mark size={3pt},
    mark options={solid,very thick},
    discard if not={threads}{32},
  ] table [
    col sep=comma,
    x=heap,
    y=med,
  ] {rel_time_by_heap_box.csv};
\nextgroupplot
  \addplot[boxplot prepared from table={table=\reltimebyheapbox,row=16,lower whisker=lw,upper whisker=uw,lower quartile=lq,upper quartile=uq,median=med,sample size=ss,draw position=heap}, boxplot prepared] coordinates {};
  \addplot[boxplot prepared from table={table=\reltimebyheapbox,row=17,lower whisker=lw,upper whisker=uw,lower quartile=lq,upper quartile=uq,median=med,sample size=ss,draw position=heap}, boxplot prepared] coordinates {};
  \addplot[
    densely dotted,
    mark=-,
    mark size={3pt},
    mark options={solid,very thick},
    discard if not={threads}{64},
  ] table [
    col sep=comma,
    x=heap,
    y=med,
  ] {rel_time_by_heap_box.csv};
\end{groupplot}
\node[right = 0.2cm of threads c1r1.east,anchor=north,rotate=90] {\num{8} Threads};
\node[right = 0.2cm of threads c1r2.east,anchor=north,rotate=90] {\num{16} Threads};
\node[right = 0.2cm of threads c1r3.east,anchor=north,rotate=90] {\num{32} Threads};
\node[right = 0.2cm of threads c1r4.east,anchor=north,rotate=90] {\num{64} Threads};\end{tikzpicture}
  \caption{Relative run-time changes to minimal memory configuration, by heap memory. The available heap memory has only a marginal impact on the run-time, with large amounts of available heap memory degrading performance independently of the number of threads used.}
  \label{fig:heap_boxplot}
\end{figure}

\subsection{Influence of Hardware Characteristics}
Based on folk knowledge about Isabelle performance,
we suspected that cache size would be a major factor;
it was debated whether boost clock speed would be relevant.
To test the hypotheses, we analyzed the impact of size of the L3-cache, base clock speed,
and maximal (boost) clock speed
(ignoring power-save cores where applicable)
on Isabelle performance.
Table \ref{tab:hardware_param_cor} shows the correlation between Isascore and those parameters (APA notation as explained in caption).
At our significance level of $0.05$,
we did not find cache size to impact performance significantly.
Base frequency is weakly correlated with the Isascore for a single thread (though at the edge of significance) and a bit more strongly (and much more significantly) in the multi-threaded scenarios.
Finally, boost frequency has a significant medium correlation for all modes,
which is strongest in the single-threaded configuration with a value of \num[round-mode=places,round-precision=2]{\cpuSpecsMulTCor}.
% !TeX root = ../main.tex


\DTLloaddb{cpuspecscor}{cpu_specs_cor.csv}

\begin{table}[htbp]
  \caption{Pearson correlation of hardware parameters with Isascore in APA notation: degrees of freedom 
(number of observations minus two),
$r$ (strength) and $p$ (significance). Cache size is left out as there is no significant correlation.}
  \label{tab:hardware_param_cor}
  \begin{tabular}{rll}\toprule
    \textbf{Mode} & \textbf{Base Frequency} & \textbf{Maximum Boost Frequency}
    \DTLforeach*{cpuspecscor}{\threads=threads,\corbase=corbase,\pbase=pbase,\nbase=nbase,\cormax=cormax,\pmax=pmax,\nmax=nmax}{
        \DTLiffirstrow{\\\cmidrule{1-3}}{\\}
        \threads & \ifdefempty\nbase{}{\cor{\nbase}{\corbase}{\pbase}} & \ifdefempty\nmax{}{\cor{\nmax}{\cormax}{\pmax}}
    }
    \\\bottomrule
  \end{tabular}
\end{table}

A possible explanation is that boost frequency can  only be sustained for a single core in most CPUs,
hence single-threaded performance profits from it a lot;
in the multi-threaded scenario, the actual core frequency is much closer to the base frequency
and thus its impact is larger.
 

\subsection{Comparison to Computational Benchmarks}
Performance benchmarks exist for many applications;
additionally, synthetic benchmarks are often used to evaluate hardware performance.
They can roughly be categorized into scientific computing versus consumer benchmarks. 
In the following, we compare the results of our Isabelle community benchmark with a number of publicly available datasets for such benchmarks.
For the comparison, we selected results with matching processors,
and matched the benchmarks' multi-thread setting (e.g., specific thread count, or all cores).
To obtain sufficiently large datasets,
we selected some of the most popular benchmarks.

\subsubsection{Benchmarks in High-Performance Computing}
The first analysis we wish to conduct is a comparison of Isabelle performance with some scientific programs.
For this analysis, we chose to import data from the High Performance Computing suite on OpenBenchmarking.
We selected the three benchmarks that had the most public results available
(in their primary configuration)
at the time of writing: \emph{Himeno}, \emph{NAMD}, and \emph{Dolfyn}.
Himeno\footnote{Results from \url{https://openbenchmarking.org/test/pts/himeno}} is an incompressible fluid analysis code written in Fortran and C~\cite{himeno}.
While a distributed memory parallel version exists (using MPI with Fortran VPP),
we concern ourselves with the sequential implementation.
NAMD\footnote{Results from \url{https://openbenchmarking.org/test/pts/namd}} is a shared memory parallel molecular dynamics code based on C++ and Charm++~\cite{namd}.
The data we use stems from machine-wide parallel trials.
Finally, Dolfyn\footnote{Results from \url{https://openbenchmarking.org/test/pts/dolfyn}} is a sequential computational fluid dynamics code based on Fortran~\cite{dolfyn}.

% !TeX root = ../main.tex

\begin{figure}[htbp]
\centering
\begin{tikzpicture}
\begin{groupplot}[
  group style={
    group name=hpcbenchmarks,
    group size=3 by 1,
    x descriptions at=edge bottom,
    y descriptions at=edge left,
    horizontal sep=55pt,
    every plot/.style={
    ymin=0,
    xmin=0,
    width=0.31\textwidth,
    height=192pt,
    }
  },
]

% himeno
\nextgroupplot[
   ylabel={Isascore (Single-threaded)},
   xlabel={MFLOPS},
   xmax=9900,
   xticklabel={\pgfmathprintnumber{\tick}},
   xtick distance=3000,
  ]
  \addplot[
    only marks,
    opacity=0.5,
  ] table [
    col sep=comma,
    x=score,
    y=isascore,
  ] {himeno_1t.csv};
  \addplot [
    thick,
  ] table [
    col sep=comma,
    x=score,
    y={create col/linear regression={x=score,y=isascore}},
  ] {himeno_1t.csv};
  \node[anchor=north east] at (rel axis cs:0.75,0.95) {$R^2=\num[round-mode=places,round-precision=2]{\himenoOneTRSq}$};
  
% dolfyn
\nextgroupplot[
    ylabel={Isabelle Time (Single-threaded)},
      y unit={\si{\minute}},
      ytick distance=720,
      ymin=0,
      yticklabel={\pgfmathparse{int(\tick/60)/100}\pgfmathprintnumber[skip 0.=true, dec sep={}, fixed,precision=2,zerofill]{\pgfmathresult}:\pgfmathprintnumber[skip 0.=true, dec sep={}, fixed,precision=2,zerofill]{0.0}},
        x unit={\si{\second}},
    xlabel={Execution Time},
    xticklabel={\pgfmathprintnumber{\tick}},
  ]
  \addplot[
    only marks,
    opacity=0.5,
  ] table [
    col sep=comma,
    x=score,
    y=time
  ] {dolfyn_1t.csv};
  \addplot [
    thick,
  ] table [
    col sep=comma,
    x=score,
    y={create col/linear regression={x=score,y=time}},
  ] {dolfyn_1t.csv};
  \node[anchor=north east] at (rel axis cs:0.75,0.95) {$R^2=\num[round-mode=places,round-precision=2]{\dolfynOneTRSq}$};
  
% NAMD vs mult-isabelle
\nextgroupplot[
ylabel={Isabelle Time (Multi-threaded)},
      y unit={\si{\minute}},
      ytick distance=360,
      ymin=0,
      yticklabel={\pgfmathparse{int(\tick/60)/100}\pgfmathprintnumber[skip 0.=true, dec sep={}, fixed,precision=2,zerofill]{\pgfmathresult}:\pgfmathprintnumber[skip 0.=true, dec sep={}, fixed,precision=2,zerofill]{0.0}},
    xlabel={Time per Simulated \si{\nano\second}},
    x unit=\si{\day},
  ]
\addplot[
    only marks,
    opacity=0.5,
  ] table [
    col sep=comma,
    x=score,
    y=time,
  ] {namd_mt.csv};
  \addplot [
    thick,
  ] table [
    col sep=comma,
    x=score,
    y={create col/linear regression={x=score,y=time}},
  ] {namd_mt.csv};
  \node[anchor=north east] at (rel axis cs:0.75,0.95) {$R^2=\num[round-mode=places,round-precision=2]{\namdMulTRSq}$};
\end{groupplot}
\node[above = 0.0cm of hpcbenchmarks c1r1.north] {Himeno};
\node[above = 0.0cm of hpcbenchmarks c2r1.north] {Dolfyn};
\node[above = 0.0cm of hpcbenchmarks c3r1.north] {NAMD};
\end{tikzpicture}
  \caption{Comparison of single-threaded (Himeno and Dolfyn) and multi-threaded (NAMD) Isabelle results with different high-performance computing benchmarks, with linear regression line.}
  \label{fig:hpc_benchmarks}
\end{figure}

\autoref{fig:hpc_benchmarks} shows the results when correlating each of the high-performance computing benchmarks with Isabelle performance.
Himeno reports performance in terms of work done over time
(where higher is better),
while NAMD and Dolfyn measure time
(per simulated \si{\nano\second}, and to solution;
lower is better).
For Himeno, we therefore compare against Isascore,
while with NAMD and Dolfyn we compare against our observed wall clock time.

NAMD, as the only benchmark of these three
that scales well with parallel resources,
has no significant correlation with single-threaded Isabelle time.
However, it has a strong linear relation with multi-threaded time.
The two less scalable benchmarks correlate much closer with Isabelle single-thread performance, where Dolfyn has a particularly nice correlation that holds well for the most performant processors.
In both cases, correlation with multi-threaded Isabelle results is much worse
($R^2$-values: Himeno \num[round-mode=places,round-precision=2]{\himenoMulTRSq}, Dolfyn \num[round-mode=places,round-precision=2]{\dolfynMulTRSq}).
For both the Isabelle benchmark and Dolfyn,
the top processor that was tested is the same
(Intel i7-12700K),
and on both benchmarks it has a margin on the runner-ups.
This is also visible on the Himeno benchmark, where the 12700K produces the highest floating point throughput of all tested processors.
However, it is not a highly parallel processor, which is why its NAMD results are less favorable.
This again shows that Isabelle performance is significantly impacted by the single-thread performance of the underlying processor.

\subsubsection{Consumer CPU Benchmarks}
For our second comparison,
we chose some of the most common consumer benchmarks to compare to:
\emph{PassMark CPU Mark}\footnote{Results from  \url{https://www.cpubenchmark.net/CPU_mega_page.html}}, \emph{Geekbench 5}\footnote{Results from \url{https://browser.geekbench.com/processor-benchmarks.json}}, \emph{Cinebench R15}\footnote{Results from \url{https://us.rebusfarm.net/en/tempbench}}, and \emph{3DMark CPU Profile}\footnote{Results from \url{https://www.3dmark.com/search}, median over the top-\num{100} values}.
For sequential performance,
\autoref{fig:consumer_benchmarks} shows the scatter plots of Isascore to consumer benchmark scores,
which are normalized to a $[0;1]$ range so the plots can be compared against.
A strong positive relationship can be observed for all benchmarks,
with $R^2$-values in the range \numrange[round-mode=places,round-precision=2]{\cinebenchOneTRSq}{\passmarkOneTRSq}.
A few moderate outliers are present (possibly due to system configuration).
All in all, the Isabelle benchmark seems quite similar to those consumer benchmarks for a single thread.
% !TeX root = ../main.tex

\begin{figure}[htbp]
\centering
\begin{tikzpicture}
\begin{groupplot}[
  group style={
    group name=benchmarks,
    group size=2 by 2,
    x descriptions at=edge bottom,
    y descriptions at=edge left,
    every plot/.style={
      width=209pt,
      height=192pt,
      ymin=0,
      ymax=170,
      xmin=-0.05,
      xmax=1.05,
    }
  },
]
\nextgroupplot
  \addplot[
    only marks,
    opacity=0.5,
    xlabel={Score},
  ] table [
    col sep=comma,
    x=score,
    y=isascore
  ] {cinebench_1t_rel.csv};
  \addplot [
    thick,
  ] table [
    col sep=comma,
    x=score,
    y={create col/linear regression={x=score,y=isascore}}
  ] {cinebench_1t_rel.csv};
  \node[anchor=north west] at (rel axis cs:0.05,0.95) {$R^2=\num[round-mode=places,round-precision=2]{\cinebenchOneTRSq}$};
\nextgroupplot
  \addplot[
    only marks,
    opacity=0.5,
  ] table [
    col sep=comma,
    x=score,
    y=isascore,
  ] {geekbench_1t_rel.csv};
  \addplot [
    thick,
  ] table [
    col sep=comma,
    x=score,
    y={create col/linear regression={x=score,y=isascore}}
  ] {geekbench_1t_rel.csv};
  \node[anchor=north west] at (rel axis cs:0.05,0.95) {$R^2=\num[round-mode=places,round-precision=2]{\geekbenchOneTRSq}$};
\nextgroupplot[
    xlabel={Normalized Score},
    every axis x label/.append style={at=(ticklabel cs:1.1)},
    ylabel={Isascore},
    every axis y label/.append style={at=(ticklabel cs:1.1)},
  ]
  \addplot[
    only marks,
    opacity=0.5,
  ] table [
    col sep=comma,
    x=score,
    y=isascore,
  ] {passmark_1t_rel.csv};
  \addplot [
    thick,
  ] table [
    col sep=comma,
    x=score,
    y={create col/linear regression={x=score,y=isascore}}
  ] {passmark_1t_rel.csv};
  \node[anchor=north west] at (rel axis cs:0.05,0.95) {$R^2=\num[round-mode=places,round-precision=2]{\passmarkOneTRSq}$};
\nextgroupplot
  \addplot[
    only marks,
    opacity=0.5,
  ] table [
    col sep=comma,
    x=score,
    y=isascore
  ] {x3dmark_1t_rel.csv};
  \addplot [
    thick,
  ] table [
    col sep=comma,
    x=score,
    y={create col/linear regression={x=score,y=isascore}}
  ] {x3dmark_1t_rel.csv};
  \node[anchor=north west] at (rel axis cs:0.05,0.95) {$R^2=\num[round-mode=places,round-precision=2]{\xdmarkOneTRSq}$};
\end{groupplot}
\node[above = 0.0cm of benchmarks c1r1.north] {Cinebench R15};
\node[above = 0.0cm of benchmarks c2r1.north] {Geekbench 5};
\node[above = 0.0cm of benchmarks c1r2.north] {PassMark CPU Mark};
\node[above = 0.0cm of benchmarks c2r2.north] {3DMark CPU Profile};
\end{tikzpicture}
  \caption{Comparison of single-threaded results with different consumer benchmarks, with linear regression line. Benchmark scores are normalized to $[0,1]$ range.}
  \label{fig:consumer_benchmarks}
\end{figure}

This gives rise to prediction of Isabelle performance,
which would allow one to judge hardware on which Isabelle was not executed before.
However, single-threaded results are not meaningful for real-world performance,
and scaling them according to the average parallel efficiency did not yield helpful results
($R^2$-values: Cinebench \num[round-mode=places,round-precision=2]{\cinebenchMulTRSq}, Geekbench \num[round-mode=places,round-precision=2]{\geekbenchMulTRSq}, PassMark \num[round-mode=places,round-precision=2]{\passmarkMulTRSq}, 3DMark \num[round-mode=places,round-precision=2]{\xdmarkMulTRSq}).
Not many datasets for consumer benchmarks report on results for different number of threads,
most report only a single \enquote{multi-core} value where all threads are utilized.
An exception to that is the 3DMark CPU Profile benchmark,
where results are reported for \numrange{1}{16} threads individually
(in steps by power of two).
This allows us to create a better correlation, because all consumer benchmarks tested had a far better parallel efficiency in the limit
and were hence not suited for direct prediction.

When using \num{8} and \num{16} threads in both the 3DMark and Isabelle benchmark,
score and Isascore are strongly to moderately correlated and have individual $R^2$-values of \num[round-mode=places,round-precision=2]{\xdmarkEightTRSq} and \num[round-mode=places,round-precision=2]{\xdmarkSixteenTRSq}, respectively.
This makes the 3DMark well suited for performance prediction.
Since the optimal number of threads is in between,
we use the average of its \num{8}-thread and \num{16}-thread results
to create a linear model for performance prediction
(tuning for a non-uniform split did not yield better results).
Using ten times ten-fold cross-validation
(i.e., averaging results over multiple iterations,
splitting the data into ten parts, and using each part as a test set and the remainder as training set),
the linear regression has an average $R^2$-value of \num[round-mode=places,round-precision=2]{\modelRSq}.
\autoref{fig:predictor} shows the final model ($R^2=\num[round-mode=places,round-precision=2]{\modelFinalRSq}$) and the resulting predictor for wall-clock time, which has a mean absolute error (MAE) of \SI[round-mode=places,round-precision=1]{\modelFinalTimeMae}{\second}.
However, that error is somewhat exaggerated by the data collection method:
The public 3DMark data only shows only the top-$100$ results, from which we use the median values.
The regression improves slightly when the
(non-public) true medians are used,
and the MAE decreases to \SI[round-mode=places,round-precision=1]{\modelPrivateFinalTimeMae}{\second}.
% !TeX root = ../main.tex

\begin{figure}[htbp]
\centering
\begin{tikzpicture}
\begin{groupplot}[
  group style={
    group name=threads,
    group size=2 by 1,
    x descriptions at=edge bottom,
    y descriptions at=edge left,
    horizontal sep=56pt,
  },
  width=0.48\textwidth,
  height=192pt,
]
  \nextgroupplot[
  xlabel={Averaged \num{8}/\num{16}-Thread 3DMark CPU Profile Score},
  every axis x label/.append style={at=(ticklabel cs:1.0)},
  ylabel={Isascore},
  ymin=0,
  xmin=0,
  xmax=9900,
  legend style={at={(0.98,0.02)},anchor=south east},
  domain=0:11000,
  ]
  \addplot[
    only marks,
    opacity=0.5,
    forget plot
  ] table [
    col sep=comma,
    x=score,
    y=isascore,
  ] {x3dmark_mt.csv};
  \addplot[thick] {\modelAX*x+\modelB};
  \node[anchor=north west] at (rel axis cs:0.05,0.95) {$R^2=\num[round-mode=places,round-precision=2]{\modelFinalRSq}$};
  \addlegendentry{%
$\num[round-mode=places,round-precision=1]{\modelB}+\num[round-mode=places,round-precision=3]{\modelAX}\cdot x$};
  \nextgroupplot[
    ylabel={Wall Clock Time},
    ytick distance=180,
    y unit=\si{\second},
    ymin=0,
    ymax=1080,
    yticklabel={\pgfmathparse{int(\tick/60)/100}\pgfmathprintnumber[skip 0.=true, dec sep={}, fixed,precision=2,zerofill]{\pgfmathresult}:\pgfmathprintnumber[skip 0.=true, dec sep={}, fixed,precision=2,zerofill]{0.0}},
    xmax=9900,
    xmin=0,
    domain=0:11000,
  ]
  \addplot[
    only marks,
    opacity=0.5,
  ] table [
    col sep=comma,
    x=score,
    y=time,
  ] {x3dmark_mt.csv};
  \node[anchor=north east] at (rel axis cs:0.95,0.95) {$\text{MAE}=\SI[round-mode=places,round-precision=1]{\modelFinalTimeMae}{\second}$};
  \addplot[thick, samples=1000] {(24*60*60) / ((\modelAX*x)+\modelB)};
\end{groupplot}
\end{tikzpicture}
  \caption{Scatter plots for final model of Isabelle performance. Linear model on Isascore on the left, corresponding wall clock run-times on the right.}
  \label{fig:predictor}
\end{figure}
The residual plot displayed in \autoref{fig:predictor_residual} has no noticeable patterns
and the residual distribution roughly follows a normal distribution.
All in all, the model simplicity and good fit
indicate that this linear model is quite well suited for performance prediction,
as long as the other system parameters are kept within reasonable bounds
and the configuration is well tuned.
% !TeX root = ../main.tex

\begin{figure}[htbp]
\centering
\begin{tikzpicture}[
    /pgfplots/scale only axis,
    /pgfplots/width=6cm,
    /pgfplots/height=4cm
]
\begin{axis}[
  name=main axis,
  ylabel={Isascore Residuals},
  xlabel={3DMark CPU Profile score},
  xmin=0,
  xmax=9900,
  ymax=200,
  ymin=-200,
]
  \addplot[
    only marks,
    opacity=0.5,
  ] table [
    col sep=comma,
    x=score,
    y=residual,
  ] {x3dmark_mt_residuals.csv};
  \draw[thick] (rel axis cs:0,0.5) -- (rel axis cs:1,0.5);
\end{axis}

\begin{axis}[
    anchor=north west,
    at=(main axis.north east),
    width=2cm,
    ytick=\empty
]
\addplot [
    hist={bins=13,handler/.style={xbar interval}},
    x filter/.code=\pgfmathparse{rawy},
    y filter/.code=\pgfmathparse{rawx},
    fill=gray!50,
] table [col sep=comma] {x3dmark_mt_residuals.csv};
\end{axis}
\end{tikzpicture}
  \caption{Residual plot for final linear model, with histogram. The residuals show no apparent patterns and roughly follow a normal distribution.}
  \label{fig:predictor_residual}
\end{figure}
