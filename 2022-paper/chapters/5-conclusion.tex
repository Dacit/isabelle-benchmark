\section{Conclusion}\label{sec:conclusion}
This work resolves our questions on Isabelle performance for the 2021-1 version.
The Isabelle community benchmark that we initiated saw lively participation
and hundreds of results were reported for a total of \num{\numCpus} distinct CPUs.
The results form a solid data base for tuning of Isabelle configuration;
when not constrained, the optimal configuration is at \numrange{8}{16} threads with \SI{16}{\giga\byte} heap memory for both the Java and ML process
(at least for HOL-Analysis, larger sessions might require more).

When buying new hardware,
the benchmark results give a good indication of which processor is desirable for Isabelle.
Individual CPU parameters are not as important
as clock speeds are only correlated with medium strength (though boost clock more than base clock)
and cache size not significantly at all.
Instead, for hardware that has not yet been tested with Isabelle,
other benchmarks can greatly help in judging performance:
While the single-threaded Isabelle benchmark score is strongly correlated with many benchmarks
(most strongly with the Dolfyn high-performance benchmark and the PassMark CPU Mark),
the multi-threaded scenario was more difficult to model;
In the end, we found a good predictor by using 3DMark CPU Profile scores from \num{8} and \num{16} threads,
with a final mean absolute error of \SI[round-mode=places,round-precision=2]{\modelFinalTimeMae}{\second}.
The model has a good fit, and one can assume that it is fairly future-proof given that hardware from the last ten years is properly predicted.
